%% ----------------------------------------------------------------
%% Background.tex
%% ---------------------------------------------------------------- 
\chapter{Background Theory}
\label{Chapter:Background}

\section{Concepts in Electrostatics and Electrodynamics}
\label{sec:concepts_in_electrostatics_and_electrodynamics}

An electric charge can be described as an excess (negative charge) or shortage (positive charge) of electrons, in comparison to a body that is electrically neutral. Such a body has equal numbers of positive and negative charges, and so has no overall charge. A charged particle in an electric field experiences a force:

\begin{equation}
 F=QE
\label{eqn:force_on_point_charge}
\end{equation}

$Q$ is the magnitude of the electric charge on the particle, and $E$ is the electric field vector. The electric field surrounding a charged particle (point charge) can also be described, using the equation below:

\begin{equation}
 E=\dfrac{1}{4\pi\epsilon}\dfrac{Q_{2}}{r^{2}}\hat{i}
\label{eqn:field_around_point_charge}
\end{equation}

$r$ is the distance from the particle centre to the point of interest, and $\hat{i}$ is the unit vector from the particle centre to the measurement location.

A dipole is a pair of opposite charges, separated by a fixed distance $d$. Dipoles can exist naturally, such as across molecules of water, or can be created by the movement of charges (see below). The net movement of electrons is described as an electric current, and the ease at which electrons move through a material defines it as a conductor, semiconductor or and insulator. Electrons are free to move through the lattice of a metallic conductor relatively unhindered, due to overlapping electron orbits. No such mechanism exists in materials described as insulators, and hence the energy required to drive electrons through such a material is significantly higher. The electrons in an insulating material are described as bound charges, because they are generally bound to an individual atom and are not free to move through the atomic lattice.

\section{Polarisation of a Dielectric Particle}
\subsection{Electronic Polarisation}

Electrons in an insulating material are considered to be bound to their parent atom, and do not readily flow through the material in the presence of an electric field. The electrons will be displaced, however, with the focus of their random orbits shifted by the force of the electric field on the charged particle. The charge in the material is no longer evenly distributed, with regions of net negative and net positive charge at either ends of the body: the material is polarised. A particle in an electric field will polarise, as charges throughout the material are displaced. The surfaces of the particle will have a net charge, as electrons are displaced either towards or away from the surfaces. This difference in net charge across the particle can be modelled as an equivalent induced dipole, and in the case of a spherical, homogenous particle in a uniform electric field, this is a valid mathematical model and is frequently used in the numerical analysis of electrokinetic systems. The polarisation of a material by the displacement of electrons is described as the electronic mechanism of polarisation.

\subsection{Interfacial polarisation}

In the case of a particle suspended in a dielectric liquid, the charges at the surface of the particle attract oppositely charged counter-charges from the liquid - this is known as interfacial polarisation. If the effective polarisability of the medium is different to that of the particle, the magnitude of the counter-charge that is developed from the medium will be different to the surface charge on the polarised particle - leading to a difference in the net charge across the particle - see Figure \ref{fig:dipole_formation_in_electric_field_diagram}.

\begin{figure}
 \centering
 \includegraphics{../Figures/dipole_formation_in_electric_field_diagram}
 % .: 1288175696x25 pixel, 105122432dpi, 31.13x0.00 cm, bb=
 \caption{A dielectric particle in a more polarisable solution: under the influence of an external electric field an equivalent dipole is formed.}
 \label{fig:dipole_formation_in_electric_field_diagram}
\end{figure}

In the case of a spherical, homogenous particle, this difference in net charge across the particle can be modelled as an equivalent induced dipole. Particles of other geometry or configuration can be modelled by an equivalent induced multi-pole. 

\subsection{Permittivity}
\subsection{Dielectric Relaxation}
\subsection{Relaxation Time}
\section{Interactions between Particles and Electric Fields}
\subsection{Electrophoresis}
As described by Equation \ref{eqn:force_on_point_charge}, a charged particle in an electric field experiences a force. A typical system of forces present on a particle suspended in a viscous fluid is shown in Figure \ref{fig:electrophoresis_forces}.

\begin{figure}
 \centering
 \includegraphics{../Figures/electrophoresis_forces.pdf}
 % .: 65855x1966111 pixel, 0dpi, infxinf cm, bb=
\label{fig:electrophoresis_forces}
\caption{The system of forces on a charged particle in an electric field.}
\end{figure}

\begin{equation}
 F_{Coulomb} + F_{Drag} = m_{p} a_{p}
\label{eqn:force_equilibrium1_electrophoresis}
\end{equation}
\begin{equation}
 F_{Coulomb} - 6 \pi r_{p} \eta v_{p} = m_{p} \dot{v}_{p}
\label{eqn:force_equilibrium2_electrophoresis}
\end{equation}

From Equation \ref{eqn:force_equilibrium2_electrophoresis} (and ignoring forces in the y-axis) it can be seen that the particle accelerates until the propulsive effect from the Coulomb force is equal to the hydrodynamic drag due to the particle motion through the viscous fluid. A system of microscopic particles in a practical fluid generally has a very low Reynold's number (see Section xxx), and so particle terminal velocity is reached almost instantaneously: a given Coulomb force effectively causes the particle to move at a velocity proportional to the magnitude of the hydrodynamic drag present on the particle. The movement of a charged particle in an electric field is known as electrophoresis.

Electrophoresis is commonly used as a scientific tool to identify small charged particles, such as proteins of DNA. Particles are identified by the distance that they move through a viscous gel under an electric field in a given time - charge and size of the particles determining the Coulomb and hydrodynamic drag forces on the particle, the sum of these determining the velocity at which the particle moves. A picture of the arrangement inside a typical machine for sample identification using electrophoresis is shown in Section 1xxx

\subsection{Dielectrophoresis}
\label{sec:bg_dielectrophoresis}
A dielectric particle placed in an electric field will polarise, forming an induced equivalent dipole. A dipole in a uniform electric field will experience a torque, directing it towards alignment with the field, but as Figure \ref{fig:dipole_in_uniform_field} shows, the net force on the dipole is zero because an equal and opposite force acts on each half of the dipole.

\begin{figure}
 \centering
 \includegraphics{../Figures/dipole_in_uniform_field.pdf}
 % d.: 257x1367 pixel, 0dpi, 6016.00xinf cm, bb=
 \caption{A dipole in a uniform field experiences a torque that directs it towards alignment with the field, but the net force on the dipole is zero.}
 \label{fig:dipole_in_uniform_field}
\end{figure}

If, however, the electric field is not uniformly distributed, the electrostatic force on each half of the dipole will be different, resulting in a net force on the dipole - this is the dielectrophoresis effect. The direction that the DEP force acts is dependent on the relationship between the polarisability of the particle and the polarisability of the medium, described by the Clausius-Mossotti factor (see below). The  time averaged DEP force can be calculated by the equation [166]:

\begin{equation}
 \langle F_{DEP} \rangle = \pi a^{3} \epsilon_{m} Re(f_{CM}) \nabla |E|^{2}
\label{eqn:dep_force}
\end{equation}

$\epsilon_{m}$ is the complex permittivity of the medium, $Re(f_{CM})$ represents the real part of the Clausius-Mossotti (CM) factor, and $E$ is the peak value of the electric field vector. The CM factor is a complex number that describes the polarisability of the system (particle and medium), and for a homogenous sphere it can be calculated as:

\begin{equation}
 f_{CM} =  \frac{\epsilon_{p}^{*}-\epsilon_{m}^{*}}{\epsilon_{p}^{*}+2\epsilon_{m}^{*}}
\label{eqn:CM_factor}
\end{equation}

$\epsilon_{p}^{*}$ is the complex permittivity of the particle, and $\epsilon_{m}^{*}$ is the complex permittivity of the medium. Complex permittivities can be calculated from the electrical properties of a material:

\begin{equation}
 \epsilon^{*} = \epsilon - j \frac{\sigma}{\omega}
\label{eqn:complex_permittivity}
\end{equation}

$\epsilon$ is the bulk permittivity of the material, $\sigma$s the conductivity, $\omega$ is the angular frequency of the applied electric field and $j$ is the imaginary vector.

The polarisation relationship between the particle and the medium define the direction of the induced dipole, and hence the DEP force. Figure \ref{fig:medoro_2007_pdep_ndep_charges} (a) shows the case where the complex permittivity of the particle is less than that of the medium - Equation \ref{eqn:CM_factor} evaluates to have a negative real part, and the DEP force vector points towards regions of lower electric field strength. This is called negative DEP. Figure \ref{fig:medoro_2007_pdep_ndep_charges} (b) shows the case where the complex permittivity of the particle exceeds that of the medium - Equation \ref{eqn:CM_factor} evaluates to have a positive real part, and the DEP force directs the particle towards regions of high electric field strength. This is called positive DEP.

\begin{figure}
 \centering
 \includegraphics{../Figures/medoro_2007_pdep_ndep_charges.png}
 % .: 1179666x1179668 pixel, 0dpi, infxinf cm, bb=
 \caption{Positive and negative DEP [Medoro 2007}
 \label{fig:medoro_2007_pdep_ndep_charges}
\end{figure}

Figure \ref{fig:CM_factor_1um_bead_0x1mS_cond} shows the variation in the Clausius-Mossotti factor for a 1 $\mu$m latex bead in an aqueous suspension of conductivity 0.1m S/m. As can be seen from Equation \ref{eqn:complex_permittivity}, the frequency is very important in determining the significance of the conductivity or permittivity. The low-frequency effects are dependent on the ratio of the conductivities, whereas the high frequencies are almost solely dependent on the ratio of the permittivities. The \textit{crossover} frequency is the point at which the CM factor is equal to zero, and occurs when:

\begin{equation}
 \epsilon_{p}^{*} = \epsilon_{m}^{*}
\label{eqn:crossover_equation}
\end{equation}

Measurement of the crossover frequency is a quick and simple method to asses the polarisability of a particle in comparison to the medium.

\begin{figure}
 \centering
 \includegraphics{../Figures/CM_factor_1um_bead_0x1mS_cond.pdf}
 % .: 1179666x1179666 pixel, 0dpi, infxinf cm, bb=
 \caption{Plot of the Clausius-Mossotti factor for a 1$\mu$m diameter latex bead in aqueous solution of conductivity 0.1m S/m.}
 \label{fig:CM_factor_1um_bead_0x1mS_cond}
\end{figure}

\subsection{Electrorotation and Travelling-wave DEP}
\label{sec:electrorotation_and_travelling_wave}

Figure \ref{fig:4_phase_sinusoidal_waveforms} shows the waveform of four alternating sinusoidal voltages, each with a phase-lag of $90^{\circ}$ to the previous wave.

\begin{figure}
 \centering
 %\includegraphics{../Figures/4_phase_sinusoidal_waveforms.pdf}
 % .: 1835208x196676 pixel, 0dpi, infxinf cm, bb=
 \caption{4-phase waveforms used to drive an electrorotation array.}
 \label{fig:4_phase_sinusoidal_waveforms}
\end{figure}

When applied to the quadrapole electrode array shown in Figure \ref{fig:polynomial_electrodes_ROT}, a rotating electric field is produced. This will interact with an induced dipole on a particle within the electrodes to produce a torque on the particle. The torque can be calculated using the equation:

\begin{equation}
 \Gamma_{ROT} = -4 \pi \epsilon_{m} a^{3} Im[CM] |E|^{2}
\label{eqn:ROT_torque}
\end{equation}


$Im[CM]$ represents the imaginary part of the CM factor. The particle will rotate either with or against the direction of rotation of the field, depending on if the imaginary part of the CM factor is positive or negative.

\begin{figure}
 \centering
 %\includegraphics{../Figures/polynomial_electrodes_ROT.pdf}
 % .: 141355216x141182464 pixel, 0dpi, infxinf cm, bb=
 \caption{Polynomial electrodes suitable for electrorotation.}
 \label{fig:polynomial_electrodes_ROT}
\end{figure}

Similarly, if the 4-phase electric fields are applied to the travelling-wave array shown in Figure \ref{fig:TWDEP_electrodes}, the field maxima will appear to move along the array, and will interact with the dipole induced on a particle above the electrodes to produce forces on the particle both perpendicular and parallel to the surface of the array.

\begin{figure}
 \centering
 %\includegraphics{../Figures/TWDEP_electrodes.pdf}
 % .: 0x0 pixel, 0dpi, nanxnan cm, bb=
 \label{fig:TWDEP_electrodes}
 \caption{An array of travelling wave electrodes constructed with multi-layer fabrication}
\end{figure}

The travelling wave force developed on a particle can be calculated from [169]:

\begin{equation}
 F_{TWDEP} = \frac{-4\pi\epsilon_{m}a^{3} Im[CM] |E|^{2}}{\lambda}
\end{equation}

$\lambda$ is the wavelength of the travelling field (determined by the geometry of the electrode array.)

A particle will experience a DEP force (proportional to the real part of the CM factor) that attracts/repels the particle from the electrodes in conjunction with a ROT or TW force. In the case of electrorotation, a positive DEP force will destabilise the electrorotation effect as the particle is drawn towards the electrodes. The phase angle of 90o between neighbouring electrodes reduces the gradient of the electric field by a factor of 2, hence reducing the DEP force on the particle by a factor of 4. For the majority of particles (that are more dense than water), a negative DEP force is required for travelling-wave manipulation so that the particle is lifted above the electrode surface. The particle will rise until an equilibrium is reached between gravity and the (vertical) DEP force. Figure \ref{fig:CM_plot_viable_yeast} shows a plot of the Clausius-Mossotti factor for viable yeast cells (S. Saccharomyces - electrical parameters from [170].) Travelling-wave manipulation is possible in the greyed regions, where the real part of the CM factor is sufficiently negative for the cells to be levitated.

\begin{figure}
 \centering
 %\includegraphics{../Figures/CM_plot_viable_yeast.pdf}
 % .: 1179666x1179666 pixel, 0dpi, infxinf cm, bb=
 \caption{Plot of the real and imaginary parts of the Clausius-Mossotti factor for viable yeast cells in aqueous solution ($\sigma_{m}$ = 50 mS/m, $\sigma_{membrane}$ = 0.25 $\mu$S/m, $\sigma_{cytoplasm}$ = 0.3 S/m, $\epsilon_{r,medium}$ =78, $\epsilon_{r,memrane}$= 6, $\epsilon_{r,cytoplasm}$ = 60) - electrorotation is feasible in the greyed frequency bands where the CM factor has a negative real part and a non-zero imaginary part.}
 \label{fig:CM_plot_viable_yeast}
\end{figure}

\section{Interactions between Fluids and Electric Fields}
\subsection{Overview}
Electrokinetic effects are not limited to the manipulation of microparticles in suspension - interaction between the electric field and the fluid itself can also take place. Electric-field induced fluid motion is of importance because it is often produced unintentionally, when electrokinetic manipulation of microparticles was the primary intention. In such cases, it can be of help or hindrance. Electric-field induced fluid motion has also been intentionally exploited, however, as a pumping mechanism. Such techniques are extremely valuable because they permit pumping hardware to be integrated into microfluidic systems.

\subsection{The Double Layer}
\subsection{Electrothermal Flow}
The presence of an electric field within a fluid will give rise to electric current flow if the fluid has a non-zero conductivity. This leads to Joule heating of the fluid, producing a temperature gradient within the system. In the case of electrodes for electrokinetic manipulation, the heat sources can be quite localised, leading to large thermal gradients in the vicinity of the electrodes. Variation in temperature produces a corresponding variation in the conductivity of the fluid, and hence thermal input produces a conductivity gradient in the fluid. In the presence of an electric field these variations give rise to a body force on the fluid, leading to fluid flow. 

Electrothermal flow is highly dependent on the fluid bulk conductivity, as power dissipation in the fluid increases with conductivity. Electroosmotic flow (see below) is usually the more dominant effect at electric field frequencies of less than 100 kHz [166].

**Equations for electrothermal flow**


\subsection{AC Electroosmosis}
Charged electrodes in a solution generate an electric field and draw counter-charges from solution towards them. Fluid flow produced by interaction between the electric field and those charges is known as electroosmotic flow. The situation is represented in Figure 2-11. Positive and negative charges (red) are drawn to the electrode surfaces in the presence of the electric field. The tangential component of the electric field acts to displace these particles near to the surface, causing fluid motion. Displacement of fluid from between the electrodes causes fluid to be drawn vertically downwards between the electrodes towards the surface. This has the effect of producing swirls of liquid over each electrode as fluid is recirculated. Reversal of the field polarity (in the case of an alternating electric field) draws opposite charges to each electrode, which subsequently experience force in the same direction so the direction of fluid flow is unchanged. The effect of AC electroosmosis is often seen while performing pDEP trapping of microparticles. Particles are drawn to the high field regions at the edges of electrodes, but are pushed back, on top of the electrodes by the fluid flow.

\begin{figure}
 \centering
 \includegraphics{../Figures/Electroosmosis_overview_diagram.pdf}
 % .: 141396832x141243464 pixel, 0dpi, infxinf cm, bb=
 \caption{Electroosmotic flow around an electrode pair}
 \label{fig:Electroosmosis_overview_diagram}
\end{figure}

\section{Electrical Characteristics of Biological Cells}
\label{sec:electrical_characteristics_of_biological_cells}
The developed internal structure of biological cells means that they have a complicated response to electrical stimulus. The most common approach to simulate the electrical characteristics of cells is to use a concentric shell model. The cell is treated as a spherical particle, with one or more discrete shells, each with uniform electrical parameters - see Figure \ref{fig:cell_single_shell_model_diagram}.

... model is derived from its physical structure... membrane, wall, cytoplasm

\begin{figure}
 \centering
 %\includegraphics{cell_single_shell_model_diagram.pdf}
 % .: 54567606x136497664 pixel, -1dpi, -70345776.00x-175965824.00 cm, bb=
 \caption{A single concentric shell model, typically used for modelling a biological cell}
 \label{fig:cell_single_shell_model_diagram}
\end{figure}

The Clausius-Mossotti factor for the single-shell model particle in suspension can be calculated by determining first the Clausius-Mossotti factor for the particle itself [171]:

Equations: Shell calculations
\begin{equation}
 f_{CM12}=\frac{\epsilon_{2}^{*}-\epsilon_{1}^{*}}{\epsilon_{2}^{*}+2\epsilon_{1}^{*}}
\label{single_shell_1}
\end{equation}

\begin{equation}
 \epsilon_{12}^{*}=\epsilon_{2}^{*}\frac{(a_{1}/a_{2})^{3}+2f_{CM12}}{(a_{1}/a_{2})^{3}-f_{CM12}}
\label{single_shell_2}
\end{equation}

\begin{equation}
 f_{CM123}=\frac{\epsilon_{12}^{*}-\epsilon_{3}^{*}}{\epsilon_{12}^{*}+2\epsilon_{3}^{*}}
\label{single_shell_3}
\end{equation}

$\epsilon_{3}^{*}$ is the complex permittivity of the medium, and can be calculated from Equation \ref{eqn:complex_permittivity}. The response of more complex particle models (with more shells) can be calculated in the same manner, by sequentially calculating the effective permittivity for pairs of shells. Figure \ref{fig:CM_plot_viable_yeast} shows a plot of the Clausius-Mossotti factor for a single shell model of S. Saccharomyces in aqueous suspension.

To model the electrical characteristics of cells, a more appropriate model maybe as a thin membrane surrounding a lossy dielectric. Small changes in the value of the cell membrane thickness in a numerical model may cause significant change in the model ouput, so it is advantageous to remove this element for a model if possible. The thickness of the membrane is much smaller than the cell radius...

We introduce the capacitance and conductance of the membrane per unit area:

\begin{equation}
 C_{m} = \frac{\epsilon_{2}}{d}  G_{m} = \frac{\sigma_{2}}{d}
\end{equation}

and can then modify the expression for the Clausius-Mossotti factor accordingly [ref Kreigmaier 2001 BBA]:

\begin{equation}
 f_{CM}=\frac{aC_{m}^{*} \left ( \epsilon_{3}^{*} - \epsilon_{1}^{*}  \right ) - \epsilon_{1}^{*}\epsilon_{3}^{*}}{aC_{m}^{*} \left ( \epsilon_{3}^{*} +2 \epsilon_{1}^{*}  \right ) +2 \epsilon_{1}^{*}\epsilon_{3}^{*}}
\label{CM_thin_membrane}
\end{equation}


\section{Flow in Microfluidic Systems}
\label{sec:flow_in_microfluidic_systems}
\subsection{Regimes of Flow}

We are all used to life in the macroscopic world, where fluids flow with turbulence and mix easily. On the micro-scale, viscous forces dominate, and inertia becomes much less important. The Reynolds number of a system describes the ratio between viscous and inertial forces:

\begin{equation}
 Re = \frac{\rho V D_{H}}{\eta} = \frac{V D_{H}}{\nu} = \frac{Q D_{H}}{\mu A}
\label{eqn:reynolds_number}
\end{equation}


$\rho$ is the fluid density, $D_{H}$ is the characteristic length of the system (or \textit{hydraulic diameter}, equal to the actual diameter in a circular pipe), $V$ is the mean fluid velocity, $\eta$ is the dynamic fluid viscosity, $A$ is the cross-sectional area and $\nu$ is the kinematic fluid viscosity. As Equation \ref{eqn:reynolds_number} shows, decreasing the length scale and the fluid velocity causes the Reynolds number to decrease. Fluid flow in a low Reynolds number system is described as laminar, and has some unique properties.

\subsection{Laminar Flow}

The flow in a microfluidic system is almost always laminar, as the dimensions of the system push the Reynolds number well below 1 - there is little lateral motion of the fluid, and the primary method of mixing is diffusion. Fluid flow follows streamlines through the channel. This means that two different liquids can flow side-by-side in a microchannel, as in Figure \ref{fig:laminar_flow_sidebyside_fluids}, which can either be of advantage or disadvantage, depending on the application.

\begin{figure}
 \centering
 %\includegraphics{../Figures/laminar_flow_sidebyside_fluids.pdf}
 % .: 1179666x1179666 pixel, 0dpi, infxinf cm, bb=
 \caption{Laminar flow in a microchannel causes different fluids to flow in streamlines with
little intermixing}
 \label{fig:laminar_flow_sidebyside_fluids}
\end{figure}

Laminar flow conditions mean that multiple fluids can be carried in the same channel without them mixing. Novel microfluidic devices have been constructed to move particles between different fluids flowing side-by-side in a microchannel, a potential application for this technology is cell lysis: cells from a blood sample are moved laterally across a microfluidic channel by DEP, into a lysis buffer, before being returned to their physiological media. Unwanted erythrocytes are destroyed, leaving the more robust leukocytes intact for further analysis. Such conditions are less than advantageous, however, for performing chemical reactions within a microchannel. Mixing of reagents by diffusion is slow, and hence a number of active and passive mixing devices have been developed to accelerate the process.

\subsection{Flow Velocity Profile}
Laminar flows exhibit streamlines, with little lateral movement or turbulence of the fluid. Hydrodynamic drag on the fluid from the walls of the channel reduces the flow rate of the fluid near to the wall, leading to clear and defined gradients in the fluid velocity across the channel. A plot of the fluid velocity over a cross-section through the channel has a parabolic profile, with zero velocity at the channel walls - see \fref{fig:2d_velocity_profile}.

\begin{figure}
	\centering
	\includegraphics{../Figures/2d_velocity_profile.pdf}
	\caption{Fluid velocity (u) profile across the vertical (y) axis of a channel as it flows in the x direction.}
	\label{fig:2d_velocity_profile}
\end{figure}

The Navier-Stokes equations describe the motion of fluids and gasses, and can be used to calculate the velocity profile across a channel if the boundary conditions can be defined. In the case of a continuous Newtonian fluid flowing under low Reynolds number conditions, the governing equation can be written as:

\begin{equation}
 \rho_{m} \frac {d\underline{u}}{dt} + \rho (\underline{u} \cdot \nabla)\underline{u} = -\nabla p + \eta \nabla^{2} \underline{u} + f
\label{eqn:simplified_navier_stokes}
\end{equation}


($\nabla p$ represents the pressure drop along the channel, $\rho$ is the density of the fluid, $\underline{u}$ is the velocity of the fluid, $\eta$ is the viscosity of the fluid and $f$ is a body force acting on the fluid.) For a 2-dimensional system (assuming the channel width along the z-direction is very large), with a steady (time invariant) flow, only along the x-axis, Equation \ref{eqn:simplified_navier_stokes} reduces to the form of:

\begin{equation}
 \frac{\partial^{2}u_{x}}{\partial y^{2}} = - \frac {p_{0}} {\eta l_{0}}
\label{eqn:2d_steady_state_navier_stokes}
\end{equation}

The differential equation can be solved using boundary conditions - namely that the
fluid velocity is zero at the channel walls (when $y=\pm d$).

\begin{equation}
 u_{x} = \frac {p_{0}} {2\eta l_{0}} (d^{2} - y^{2})
\label{2d_steady_state_solution_navier_stokes}
\end{equation}


An alternative method for calculating the flow velocity profile when the volumetric flow rate is known (for example if fluid is pumped using a positive displacement device, such as a syringe pump) is to calculate the pressure drop over the length of the channel from the volumetric flow rate.

\begin{equation}
 Q =  \int_{-d}^{d} u_{x} dy
= \frac{2}{3}\frac{d^{3}}{\eta} \frac{p_{0}}{l_{0}} 
\therefore \frac{p_{0}}{l_{0}} = \frac{3 \eta Q}{2 d^{3}}
\label{derivation_vfp_from_vfr}
\end{equation}

\begin{equation}
 \therefore u_{x} = \frac{3 Q}{2 d^{3}}(d^{2}-y^{2})
\label{2d_solution_from_Q_navier_stokes}
\end{equation}

\section{Forces Acting on a Particle Within a Microfluidic System}
\label{sec:Forces_Particle_Microfluidic_System}

Generalised model of forces in a MF system.

\subsection{Gravitational}
\subsection{Buoyancy}
\subsection{Hydrodynamic Drag}
\subsection{Brownian Motion}
\subsection{Electrokinetic}
