%% ----------------------------------------------------------------
%% Conclusions.tex
%% ---------------------------------------------------------------- 
\chapter{Conclusions} \label{Chapter: Conclusions}

Microfluidic electrokinetic devices have been used to separate cells and particles in heterogeneous populations using fluorescent signals and image-based sorting. Two methods of separation have been developed: trapping target cells within dielectrophoretic ring traps so that they can be held as other cells are removed in the fluid flow, or deflecting target particles within a fluid stream so that they are carried by the flow towards a particular outlet. Both technologies have been used to recover small sorted populations with 100\% purity.

\section{Technological Achievements}

\subsection{Image-based Particle Sorting}

Image-based particle detection significantly reduces the complexity of the optical equipment needed to collect data for cell sorting, by using a single sensor (a colour CCD camera) and moving the task of data separation into software algorithms. Conventional fluorescence-activated cell sorters interpret the signal from photomultiplier tubes as a cell passes through a detection region. Modern devices incorporate multiple lasers for illumination at different wavelengths, and obtain intensity data across a range of different wavelength bands, each with an individual sensor. Scattered light is also measured in the forward and side axes, which provides a measure of the approximate cell size and granularity respectively. We have demonstrated the simultaneous measurement of fluorescence information in two bands, and it would be simple to extend this to three or four bands with the appropriate filter set. It would be difficult to obtain as much data as is possible with a modern FACS machine using a single sensor, although it is unlikely that all of the bands would need to be used simultaneously. The use of video data permits a range of measurements to be taken, such as particle colour, size and shape, although it does place a limitation on the rate at which particles can be sorted as it cannot exceed the camera frame rate without potential loss of accuracy. 

While conventional fluorescence-based sorting devices seek to minimise the detection volume in order to increase sensitivity, with image-based sorting the detection volume can be defined dynamically. The image processing algorithms use a feature recognition and threshold function to isolate regions of the image that contain particles, so the detection region is optimised individually for each particle. While it is not possible to equal the sensitivity of a photomultiplier tube using only a video camera, this method goes some way to improving the results.

For particle detection and analysis over a large area, such as to control a large array of trapping electrodes, image-based detection is one of the few feasible choices. It would rapidly become impractical to provide a single sensor for each trap for even moderately sized arrays if standard photomultiplier tubes were used. Some success has been had integrating photosensors into DEP manipulation arrays on to silicon substrates \citep{Manaresi:2003}; alternatively a multiplexed sensor could be used.

Generation of droplets in a conventional FACS machine simplifies the sorting operation as droplets pass through the sorting electrodes at regular intervals and their spacing can be controlled. This limits the possibility of two cells entering the sorting region in close succession, and one being sorted incorrectly. It is still possible for two cells to be placed in the same droplet, however, although they will be detected simultaneously and the appropriate action taken. With the microfluidic system used in this study, particles approach the electrodes in a random, probabilistic manner. The average spacing between particles can be adjusted by controlling the particle concentration in solution, but it is still possible for two or more different particles to enter the sorting region separated by a distance that is insufficient for both to be sorted correctly. Fortunately, by using image-based sorting a large region of the microchannel can be monitored for particles without sacrificing sensitivity, and multiple particles tracked. This permits conservative sorting strategies to be implemented, and both particles can be rejected to a `waste' output to preserve the purity of the sorted population.

Image-based particle manipulation and sorting is only likely to be the preferred option for particular and specialised applications, as it is unable to match the sensitivity and speed possible with other optical detectors. This is particularly the case in the cell sorting arena, where high throughput enables the separation of even smaller minority groups. The burgeoning interest in stem cell therapies highlights the sorting problem: although it is relatively simple to isolate the mononuclear fraction of an ex vivo sample before sorting, stem cells may still make up less than 1 in 10,000 of these cells \citep{Pittenger:1999}. Unless such numbers of cells can be sorted in a reasonable length of time, it is unlikely that even a single target cell will be detected.

\subsection{Microfluidics and Electrokinetics}

Dielectrophoretic particle manipulation has been shown suitable for both trapping particles and deflecting particles at a sorting junction. Although the DEP force can be scaled up for higher throughput by increasing the electrode voltage, limitations are imposed by the onset of electrolysis in the fluid and electrothermal heating - the later being a particular problem if cells are suspended in high conductivity physiological media. 
 
Trapping single particles within a microfluidic device has many potential applications, not least as a sorting technique. It is a useful step in cell patterning, for cell culture, or as a cell concentration stage. The method also provides a means to bypass the limitation on the sorting rate imposed by the camera frame rate during image-based sorting, as multiple cells can be trapped simultaneously. The technique is most useful for applications that require cells to remain `on-chip', such as when an integrated analysis or culture stage is used, as such devices typically require small numbers of cells to be sorted or processed and the ability to concentrate cells within the microfluidic system is most useful. 

Deflection of particles as they flow through a microfluidic channel has been demonstrated as suitable for the sorting of fluorescent synthetic particles, and work is underway on the sorting of cells. Although the principle of operation is identical for the sorting of cells or synthetic particles, some optimisation must be performed if useful levels of purity are to be obtained when sorting cells, as they are a less `ideal' population, with significant variations in fluorescent intensity and size, and are more prone to the formation of aggregates. There are also some applications that would benefit from the ability to manipulate synthetic particles, such as the increasing number of chemical processes that are performed on the surface of microparticles, including DNA sequencing \citep{Hultman:1989}.

\section{Future Potential}

Cell and particle sorting is likely to remain an important laboratory process, and equipment is likely to gradually become faster and more accurate. The paucity of techniques and markers to identify some cell types places a constraint on the manner and accuracy in which cells can be sorted. Immunological tagging methods such as fluorescently-tagged monoclonal antibodies or magnetic nanoparticles remain the most popular choice, although label-free methods are receiving particular attention as research aims to avoid the use of potentially toxic markers that may limit subsequent use. Electrical impedance spectroscopy has been demonstrated as a cell analysis tool, able to differentiate between leukocytes on size and electrical properties \citep{Holmes:2007}, although as yet has not been integrated into a sorting device. A dielectrophoretic sorting gate triggered by an impedance analysis system could form a well integrated all-electric system, and would be a natural progression.

There is potential to increase the sorting rate of both the ring trap and the sorting gate systems, although neither are likely to reach the high throughput of modern commercial FACS machines. The maximum sorting rate limit imposed by the video hardware could potentially be increased by up to nine times by increasing the frame rate to the hardware maximum (90 fps). The control software would also need to be improved to match the increased speed of operation, although this would not be difficult if a lower-level programming language were used. Although even faster camera equipment is available, the photosensitivity would start to be a concern as the decreasing exposure time reduces the image intensity. More sensitive optical equipment could be used, such as a photomultiplier tube, although that would negate the advantages of using image-based sorting. The sorting rate could be increased by increasing either the concentration of particles within the device or the fluid flow rate through the device; the later would also require a proportional increase in the DEP force acting on the particles, typically achieved by increasing the electrode voltage. The ring trap design is highly scalable, and it would be a simple matter to increase the number of traps triggered from a single camera. This would increase both the particle sorting rate and the number of cells that could be sorted before the array was fully populated. 

Ultimately, as with all incremental technological development, a new technology will be adopted if it offers a clear advantage over the current state of the art. In the field of cell and particle sorting the key performance indicators are sorting rate, purity, and viability of recovered populations. As was shown in \cref{Chapter:Introduction}, however, care must be taken to examine the operating conditions under which a sorting device reaches its peak performance before an informed decision can be made. Novel technologies may be well exploited if they can find their niche application. Dielectrophoretic particle manipulation technology is capable of sorting small numbers of cells with high purity and requires little ancillary equipment, so is particularly suited to use as a preparatory stage within an integrated microfluidic system.

\section{Publications arising from this work}

Rupert S. Thomas, Hywel Morgan and Nicolas G. Green. Negative DEP traps for single cell immobilisation. \textit{Lab on a Chip}, 9:1534-1540, 2009.

Rupert S. Thomas, Peter D. Mitchell, Richard O.C. Oreffo and Hywel Morgan. Trapping single osteoblast-like human cells from a
heterogeneous population using a dielectrophoretic microfluidic device. \textit{Biomicrofluidics}, submitted.

Rupert S. Thomas, Peter D. Mitchell, Richard O.C. Oreffo and Hywel Morgan. Dielectrophoretic sorting gates for fluorescence-activated cell isolation, manuscript in preparation.