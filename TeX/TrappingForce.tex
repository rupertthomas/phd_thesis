\chapter{Characterisation of the Trapping Force of Dielectrophoretic Traps} 
\label{Chapter:Trapping_Force}


\section{Introduction}

Dielectrophoretic traps are particularly useful for the isolation and immobilization of cells. A single cell can be removed from a fluid flow and isolated from other cells in a fixed position. This enables measurements to be obtained from a single cells, such as impedance or fluorescence measurements (ref ???) and for such measurements to be expanded across the temporal range (!!!) (ref Voldman).

The ability to trap single cells as required can also be used to separate rare or important cells from a heterogenous population. Target cells can be immobilised in an array of traps, while unwanted cells are removed by fluid flow. Traping advantages: 
Confinement to highly localised spatial positioning 
Suited to measurements along the temporal dimension 
Isolation -issue with cell culture requirements for cell contact

Trap requirements: Single particle Coplanar Negative DEP Suitable for use in culture medium 
Closed trap
An ideal dielectrophoretic cell trap should have the following characteristics: 
Operate in (high conductivity) physiological media
Have minimum power dissipation (avoid fluid heating). 
Limit the exposure of cells to high electric fields 
Operate at high frequencies to minimise induced transmembrane potentials. 
Capture a single cell in a closed cage.
Be scalable to an array, ideally with a single wire connection per trap.

Figure 4.1 shows forces on a particle in a DEP trap - if the particle is immobilised then
a force equilibrium will exist:

** Figure 4.1: Forces on a particle immobilised in a DEP trap against a 
uid 
ow.

**eq: FDEP;x = FHD;x (4.1)
**eq: mg + FDEP;y = FB + FL + FN

where the force (F) subscripts correspond to: DEP - dielectrophoretic, HD - hydrodynamic (Stokes) drag, B - buoyancy, L - hydrodynamic lift, N - normal surface reaction. As was shown in Section 3.3, the electric eld strength varies greatly across the centre of the ring array, with a zero value in the centre and maximum at the electrode edges in the gap between the electrodes. At any given height above the substrate, the lateral DEP force FDEP;x is zero in the centre, and increases to a maximum over the ring electrode. Hence, a trapped particle in a fluid flow under steady-state conditions is displaced a certain distance from the centre of the ring. This position is given by the balance of the Stokes drag force and the DEP trapping force.

\section{Materials and Methods}

Electrodes 
The electrodes were fabricated on 150mm diameter, 700 $\mum$ thick glass
wafers. Electrode layers were made from layers of titanium (for adhesion) and platinum,
patterned using photolithography and ion beam milling. As mentioned previously, in
order to fabricate a ring electrode in the ground plane, two metal layers separated by
a dielectric insulator are required. The dielectric was a 1 m thick layer of benzocy-
clobutene (BCB) patterned using reactive ion etching. Wafers were diced into individual
chips, 20 mm square. The ring electrodes were fabricated with internal diameters of 40
and 80 m, the width of the ring electrode was 10 m and the gap between the ring and
the ground plane was 10 m.

Microuidic Packaging 
The microuidic channel was fabricated separately on each
chip, from a layer of Ordyl SY355 dry lm resist (Elga Europe), bonded between the
chip and a glass lid. One layer of resist was laminated on to each of the two surfaces
(chip and glass lid) by hot-rolling at 100C. The laminate was patterned by exposure
to UV radiation through a negative contact mask and developed in BMR developer
(Elga Europe) using a process similar to that described by Vulto et al [24]. A closed
micro
uidic channel was produced by bonding the two resist layers together at 200C.
Inlet and outlet holes (1mm diameter) were drilled in the glass lid after bonding.

Equipment
Microuidics 
A microuidic manifold was used to interface macroscale fluidic connections to the microdevice and also provided electrical contact via spring contacts
mounted on a printed circuit board. Bead suspension was driven through the device
using a Cole-Palmer 79000 syringe pump with 
ow rates in the range 0.25 to 20 l/min.
Electronics
Microscopic Observations Particles were imaged and tracked using a home-made

uorescence microscope, built around a Nikon 10x Plan Fluor objective lens and a Pana-
sonic AW-E600E colour camera. A blue LED (Lumiled Luxeon, peak output 470nm) pro-
vided illumination for (FITC/GFP compatible) 
uorescence observations, while broad-
band illumination from a 'white' LED (5500K CCT) mounted underneath the target
was used for transmitted-light measurements.

Cells and Microparticles
Synthetic Microparticles Latex test particles were suspended in a solution of
0.1mM KCl containing 0.02% (v/v) TWEEN-20, prepared in deionised water. The
conductivity was measured to be 1.9mS/m (25C) using a (Hanna EC215) conductivity
meter. Polystyrene microspheres (Polybeads, Polysciences Ltd) were purchased from
Park Scientic Inc, and had mean diameter of 15.61?m (CV  15%, density 1.05.). For
trap characterisation, a 100l aliquot of bead suspension (1:35  107 beads/ml, or 2.5%
solids) was washed three times in the 0.1mM KCl/TWEEN solution by centrifugation
and resuspension. Bead solutions were passed through a 41m lter (Whatman) prior
to use.
Cells and Culture GFP-modied HeLa (Human epithelial carcinoma) emit green

uorescence when illuminated with light in the 450-500nm region. Cells were cultured
in DMEM (Dulbecco's Modied Eagle's Medium - 4mM L-glutamine, Hepes buer, no
Pyruvate) with 10% foetal calf serum and 100 g/ml Penicillin/Streptomycin, at 37C. To
maintain growth, the cultures were split every 3rd or 4th day by trypsinisation, and fresh
culture medium added. For experiments, the cells were removed from culture, incubated
at 37C and used within 3 hours. The cells were concentrated by centrifugation in culture
medium to a density of 106 cells/ml. Prior to use, the chip was 
ushed through with
DMEM, and a sample of HeLa cell suspension injected at a 
ow rate of 10 l/min.


**CM factor plot of cells in high conductivity media

Figure 4.2: Plot of the real and imaginary parts of the Claussius-Mossotti factor for
15m polystyrene spheres (r;p = 2:5, Ks = 1  10..9) in aqueous solution (r;m = 78,
m = 1:9mS=m)

Characterisation of Trapping Force

A bead suspension was pumped through the channel, and a single bead immobilised in
a ring trap using a signal of 1MHz at 5V pk-pk. With a bead trapped, the 
ow rate
was increased in steps from 0 to 5.5 ll/min, and the position of the bead recorded.
Data was recorded for 10 seconds for each 
ow rate, and 20 frames from each clip at
0.5 second intervals were analysed. Bead position relative to the centre of the trap was
measured (in pixels, and converted to m) for each frame, and an average value for all 20
frames was obtained. This experiment was repeated four times. The data is plotted in
Figure 4a, showing bead displacement against volumetric 
ow rate. The displacement
from the centre of the array increases with increasing 
ow rate, but the rate of increase
slows as the particle approaches the ring due to the rapidly increasing DEP force. At
an applied voltage of 5V pk-pk, the beads escaped from the trap when the 
ow rate
exceeded 5.5 ?l/min.
For comparison with experimental data, the force on a 15.6 m diameter polystyrene par-
ticle was calculated using Equation 2.5 and the simulated electric eld, setting Re(fCM)
= -0.475 (with r =2.5, p=0.27mS/m, V = 5V pk-pk and f = 1 MHz). Only the hor-
izontal component of the DEP force is considered, as this is the only component that
can be determined directly from the hydrodynamic drag force.

\section{Results}

Single 15m polystyrene microparticles were immoilised from a 
uid 
ow and trapped
in the 80m diameter ring traps - Figure 4.3a. Metallised regions re
ect episcopic
illumination, so appear blue. Transparent regions appear green due to white diascopic
illumination through the GFP 
uorescence lter set.
Once energised, the trap was closed and surrounding particles were de
ected around
or over the trap. Particles could be held within the trap at 
ow rates of up to 5.5
l min..1 (with electrical excitation of 1 MHz, 2.5Vp), after which they were displaced
from the trap by hydrodynamic drag. The particles were also held in an array of 8
40m diameter traps - (Figure 4.3b). The smaller size produced stronger DEP forces,
and particles could be held within the trap at 
ow rates of up to 20 l min..1
Single HeLa cells (suspended in DMEM medium, m = 1:6S=m) were also trapped in
the ring traps - Figure 4.4. In the 40m diameter traps the cells could be held against

ow rates of up to 1.5 l min..1 (with electrical excitation of 20 MHz, 10Vp).
Trapping forces were lower because of:.... CM factor Double layer -> voltage drop Signal
attenuation at HF
Higher frequencies were used as damage to the electrodes was observed at 1 MHz. more
on electrode damage...

Figure 4.3: An array of four ring traps each with a single polystyrene microparticle
trapped inside.

Figure 4.4: A single HeLa cells (Cervical cancer, GFP-modied) immobilised in a ring
trap. Suspended in DMEM culture medium, electrical excitation is 20Vpp, 20MHz.

Figure 4.5: 15 m diameter polystyrene beads were trapped in the centre of a ring
trap (electrical excitation 2.5 V peak, 1 MHz) but were increasingly displaced by hy-
drodynamic drag as the volumetric 
ow rate of the 
uid was increased.

Figure 4.6: The trapping force developed by the ring traps on 15 m diameter
polystyrene beads was calculated from models of the 
uid velocity prole, and was
found to be similar to values derived from FEA simulations of the electric eld distri-
bution. The 
uid velocity, at a distance from the channel wall equal to the particle
radius, is shown on the alternate axis.

Figure 4.7: Data from numerical simulations ts closely with experimental results
(R2=0.9958) is obtained if the applied voltage used in the simulation is reduced to 4.8V
pk-pk, suggesting that a small voltage drop could have occurred along the interconnects.

\section{Discussion}

The displacement from the centre of the array increases with increasing 
ow rate, but
the rate of increase slows as the particle approaches the ring due to the rapidly increasing
DEP force. At an applied voltage of 5V pk-pk, the beads escaped from the trap when
the 
ow rate exceeded 5.5 l/min.
The results are plotted as a line in Figure ??. Also shown are the values of the trap-
ping force determined from the experimental data, assuming that it is balanced by the
modied Stokes drag force, calculated using equation 6 and the volumetric 
ow rate.
The velocity of the 
uid is shown on the opposite axis; as the particle is in a shear 
ow
this is the velocity impinging on the centre of the particle. Comparison of the experi-
mental data with the simulated force shows excellent agreement, with small deviations
in the centre and edge of the trap. The discrepancy at small displacements may be
due to errors in measurement of small displacements and the diculty in controlling
low 
ow rates. At the edge of the trap the error may be due to the limitations of the
dipole approximation used to calculate the force. Equally, a near perfect agreement
(R2=0.9958) is obtained if the applied voltage used in the simulation is reduced to 4.8V
pk-pk, suggesting that a small voltage drop could have occurred along the interconnects.
The maximum trapping force developed on a 15.6 m latex bead was 23 pN, sucient
to immobilise the particle against a 
ow of 5.5 l/min. To put this into context, this
exceeds the particle's weight force of 20.48 pN (assuming density = 1.05 g/ml, particle
mass = 2.09 x 10..12 kg). Hence, in the absence of a 
ow, the particle would remain
trapped if the trap array were to be held vertically. The DEP trapping force scales
with the third power of particle radius (a3, equation 1), while the hydrodynamic drag
scales with the rst power of radius (equation 4). This means that larger particles
can be trapped at higher 
ow rates than smaller particles, for a given value of applied
voltage and trap size. The maximum 
ow rate against which biological cells can be held
is generally lower than for similarly sized polystyrene particles, because the Clausius
Mossotti factor for cells suspended in physiological media is lower than for polystyrene
particles at frequencies suitable for nDEP.

Trapping force Comparison with FEA Reasons for t error

Second trap location at roof of channel

Connectivity issues: Arrayed operation - matrix control
Open vs closed traps Switching on - auto control

\section{Conclusions}

\subsection{subsection title}
\subsection{subsubsection title}
\paragraph{paragraph title}
\subparagraph{subparagraph title}

